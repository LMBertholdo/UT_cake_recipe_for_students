\section{Related Work}

\begin{enumerate}
\item Go to Google scholar and search using \textbf{set of keywords} related to your research. How many entries resulted from those sets? put this in a excel sheet.
\item (For each set of keywords) copy and paste the top 20 titles of papers (from Google Scholar) in a Excel. Judge whether the title has something similar to the work that you intend to do. Update the excel file.
\item From the similar/interesting papers (based-on-title), read their abstracts. In the Excel file mark which papers have interesting abstracts. 
\item Each paper with interesting titles and abstracts, you should read the introduction and mark on Excel which have an interesting introduction.
\item Papers with interesting titles, abstracts, and introduction, you should read the conclusion and mark on Excel which have an interesting conclusion.
\item The papers that are interesting till this point you should read the entire paper! Update the Excel what are the similarities with the other papers and what are the differences.
\item Create a table and place it in the 'Related Work' section of your proposal/paper/thesis. This table should contain the reference to the paper, the similarities and the differences. 
\end{enumerate}

For the phase after the proposal you should re-read those interesting papers and take note of their "related work" papers. Read those papers and update the Excel Sheet.

\textbf{NOTE: that the final goal of this section is a table that summarizes the characteristics of each paper and your critical analysis to highlight the existing gaps of research.}

Examples of how to make a reference:
\begin{itemize}
	\item $\backslash$citep outputs: \citep{jjsantanna2015IM1}
	\item $\backslash$citet outputs: \citet{jjsantanna2015IM1}

\end{itemize}
